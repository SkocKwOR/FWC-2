\def\mytitle{Assignment Probability}
\def\myauthor{P PAVAN KUMAR}
\def\contact{padmanabhunipavan0@gmail.com}
\def\mymodule{Future Wireless Communication (FWC)}
\documentclass[10pt, a4paper]{article}
\usepackage[a4paper,outer=1.5cm,inner=1.5cm,top=1.75cm,bottom=1.5cm]{geometry}
\usepackage[parfill]{parskip}
\usepackage{lmodern}
\usepackage{tikz}
\usepackage{physics}
\usepackage{tabularx}
\usepackage{enumitem}
\usetikzlibrary{calc}
\usepackage{amsmath}
\usepackage{amssymb}
\renewcommand*\familydefault{\sfdefault}
\usepackage{lipsum}
\usepackage{xcolor}
\usepackage{listings}
\usepackage{float}
\usepackage{titlesec}
\providecommand{\mtx}[1]{\mathbf{#1}}
\titlespacing{\subsection}{1pt}{\parskip}{3pt}
\titlespacing{\subsubsection}{0pt}{\parskip}{-\parskip}
\titlespacing{\paragraph}{0pt}{\parskip}{\parskip}
\providecommand{\qfunc}[1]{\ensuremath{Q\left(#1\right)}}
\providecommand{\sbrak}[1]{\ensuremath{{}\left[#1\right]}}
\providecommand{\lsbrak}[1]{\ensuremath{{}\left[#1\right.}}
\providecommand{\rsbrak}[1]{\ensuremath{{}\left.#1\right]}}
\providecommand{\brak}[1]{\ensuremath{\left(#1\right)}}
\providecommand{\lbrak}[1]{\ensuremath{\left(#1\right.}}
\providecommand{\rbrak}[1]{\ensuremath{\left.#1\right)}}
\providecommand{\cbrak}[1]{\ensuremath{\left\{#1\right\}}}
\providecommand{\lcbrak}[1]{\ensuremath{\left\{#1\right.}}
\providecommand{\rcbrak}[1]{\ensuremath{\left.#1\right\}}}
\providecommand{\pr}[1]{\ensuremath{\Pr\left(#1\right)}}
\newcommand*{\permcomb}[4][0mu]{{{}^{#3}\mkern#1#2_{#4}}}
\newcommand*{\perm}[1][-3mu]{\permcomb[#1]{P}}
\newcommand*{\comb}[1][-1mu]{\permcomb[#1]{C}}
\newcommand{\myvec}[1]{\ensuremath{\begin{pmatrix}#1\end{pmatrix}}}
\let\vec\mathbf
\lstset{
frame=single, 
breaklines=true,
columns=fullflexible
}
\def\inputGnumericTable{}
\usepackage[latin1]{inputenc}
\usepackage{fullpage}
\usepackage{color}
\usepackage{array}
\usepackage{longtable}
\usepackage{calc}
\usepackage{multirow}
\usepackage{hhline}
\usepackage{ifthen}
\title{\mytitle}
\author{\myauthor\hspace{1em}\\\contact\\FWC22088\hspace{6.5em}IITH\hspace{0.5em}\mymodule\hspace{6em}probability}
\begin{document}
	\maketitle
\section{Problems}
\begin{enumerate}
\item Q:11,16.4,4
\begin{enumerate}
\item one ticket
\item two tickets
\item 10 tickets
\end{enumerate}
\end{enumerate}
\subsection{Problem}
\paragraph{Q1: In a certain lottery 10,000 tickets are sold and ten equal prizes are awarded.What is the probability of not getting a prize if you buy (a) one ticket (b) two tickets (c) 10 tickets ?\\}	

\textbf{Solution:}\\
\begin{table}[h]
	\input{tables/table.tex}
	\caption{variable description}\label{Table1}
\end{table}
\\Total number of possible outcomes = $\comb{N}{n}$\\
Total number of favourable outcomes = $\comb{q}{n}$\\
Probability $\pr{n} = \cfrac{\comb{q}{n}}{\comb{N}{n}}$ \\
\subsubsection{a : one ticket}
\begin{align}
Probability = \pr{n=1}= \frac{\comb{9990}{1}}{\comb{10000}{1}} = 0.9990
\end{align}
\subsubsection{b : two ticket}
\begin{align}
Probability = \pr{n=2} = \frac{\comb{9990}{2}}{\comb{10000}{2}} = 0.9980
\end{align}
\subsubsection{c : 10 ticket}
\begin{align}
Probability = \pr{n=10} = \frac{\comb{9990}{1}}{\comb{10000}{1}} = 0.9901
\end{align}
\end{document}
