\def\mytitle{Assignment Probability}
\def\myauthor{P PAVAN KUMAR}
\def\contact{padmanabhunipavan0@gmail.com}
\def\mymodule{Future Wireless Communication (FWC)}
\documentclass[10pt, a4paper]{article}
\usepackage[a4paper,outer=1.5cm,inner=1.5cm,top=1.75cm,bottom=1.5cm]{geometry}
\twocolumn
\usepackage[parfill]{parskip}
\usepackage{lmodern}
\usepackage{tikz}
\usepackage{physics}
\usepackage{tabularx}
\usepackage{enumitem}
\usetikzlibrary{calc}
\usepackage{amsmath}
\usepackage{amssymb}
\renewcommand*\familydefault{\sfdefault}
%\usepackage{watermark}
\usepackage{lipsum}
\usepackage{xcolor}
\usepackage{listings}
\usepackage{float}
\usepackage{titlesec}
\providecommand{\mtx}[1]{\mathbf{#1}}
\titlespacing{\subsection}{1pt}{\parskip}{3pt}
\titlespacing{\subsubsection}{0pt}{\parskip}{-\parskip}
\titlespacing{\paragraph}{0pt}{\parskip}{\parskip}
\providecommand{\qfunc}[1]{\ensuremath{Q\left(#1\right)}}
\providecommand{\sbrak}[1]{\ensuremath{{}\left[#1\right]}}
\providecommand{\lsbrak}[1]{\ensuremath{{}\left[#1\right.}}
\providecommand{\rsbrak}[1]{\ensuremath{{}\left.#1\right]}}
\providecommand{\brak}[1]{\ensuremath{\left(#1\right)}}
\providecommand{\lbrak}[1]{\ensuremath{\left(#1\right.}}
\providecommand{\rbrak}[1]{\ensuremath{\left.#1\right)}}
\providecommand{\cbrak}[1]{\ensuremath{\left\{#1\right\}}}
\providecommand{\lcbrak}[1]{\ensuremath{\left\{#1\right.}}
\providecommand{\rcbrak}[1]{\ensuremath{\left.#1\right\}}}
\newcommand*{\permcomb}[4][0mu]{{{}^{#3}\mkern#1#2_{#4}}}
\newcommand*{\perm}[1][-3mu]{\permcomb[#1]{P}}
\newcommand*{\comb}[1][-1mu]{\permcomb[#1]{C}}
\newcommand{\myvec}[1]{\ensuremath{\begin{pmatrix}#1\end{pmatrix}}}
\let\vec\mathbf
\lstset{
frame=single, 
breaklines=true,
columns=fullflexible
}

\title{\mytitle}
\author{\myauthor\hspace{1em}\\\contact\\FWC22088\hspace{6.5em}IITH\hspace{0.5em}\mymodule\hspace{6em}probability}
\begin{document}
	\maketitle
\section{Problem}
\begin{enumerate}
\item Q:12,13.4,4
\end{enumerate}
\subsection{Problem}
\paragraph{Find the probability distribution of\\
(i) number of heads in two tosses of a coin.\\
(ii) number of tails in the simultaneous tosses of three coins.\\
(iii) number of heads in four tosses of a coin.\\\\\\}
\textbf{solution:}\\
\textbf{(i) number of heads in two tosses of a coin.}\\
Given, number of trails = n = 2\\
probability of getting head for one coin = p =$\frac{1}{2}$\\
probablity of not getting a head = q = 1-p = $\frac{1}{2}$\\
let X represent the number of heads in two tosses of a coin
$\therefore$ the values of X = \{0,1,2\} \\
 by using binomial distribution\\
 \begin{align*}
 P(X) = \comb{n}{X}p^Xq^{n-X}
 \end{align*}
Thus,the required probility distribution is\\
\begin{center}
\begin{tabular}{ |c|c|c|c| }
\hline
X & 0 & 1 & 2 \\
\hline
P(X) & $\frac{1}{4}$ & $\frac{1}{2}$ & $\frac{1}{4}$\\
\hline
\end{tabular}\\
\vspace{2mm}
\end{center}
\textbf{(ii) number of tails in the simultaneous tosses of three coins.}\\
Given, number of trails= n = 3\\
probability of getting tail for one coin = p = $\frac{1}{2}$\\
probalility of not getting tail =  q = 1-p = $\frac{1}{2}$\\
let X represents the number of tails in simultaneous tosses of three coins
 $\therefore$ the values of X = \{0,1,2,3\}\\
 by using binomial distribution\\
 \begin{align*}
 P(X) = \comb{n}{X}p^Xq^{n-X}
 \end{align*}
Thus,the required probility distribution is\\
\begin{center}
\begin{tabular}{ |c|c|c|c|c| }
\hline
X & 0 & 1 & 2 & 3\\
\hline
P(X) & $\frac{1}{8}$ & $\frac{3}{8}$ & $\frac{3}{8}$ & $\frac{1}{8}$\\
\hline
\end{tabular}\\
\end{center}
\textbf{(iii) number of heads in four tosses of a coin.}\\
given, number of trails n = 4\\
probability of getting a head for one coin = p =$\frac{1}{2}$\\
probability of not getting a head = q = 1-p = $\frac{1}{2}$\\
let X represents the number of tails in simultaneous tosses of three coins
 $\therefore$ the values of X = \{0,1,2,3,4\}\\
 by using binomial distribution\\
 \begin{align*}
 P(X) = \comb{n}{X}p^Xq^{n-X}
 \end{align*}
Thus,the required probility distribution is\\
\begin{center}
\begin{tabular}{ |c|c|c|c|c|c| }
\hline
X & 0 & 1 & 2 & 3 & 4\\
\hline
P(X) & $\frac{1}{16}$ & $\frac{4}{16}$ & $\frac{6}{16}$ & $\frac{4}{16}$ & $\frac{1}{16}$\\
\hline
\end{tabular}\\
\end{center}
\end{document}